\documentclass[dcc,uchile]{fcfmcourse}
\usepackage{teoria}
\usepackage[utf8x]{inputenc}
\usepackage{amsmath}
\usepackage{amsfonts,setspace}
\usepackage{listings}
\usepackage{hyperref}
\usepackage{color}





\definecolor{pblue}{rgb}{0.13,0.13,1}
\definecolor{pgreen}{rgb}{0,0.5,0}
\definecolor{porange}{rgb}{0.9,0.5,0}
\definecolor{pgrey}{rgb}{0.46,0.45,0.48}

\lstset{language=Java,
  showspaces=false,
  showtabs=false,
  breaklines=true,
  showstringspaces=false,
  breakatwhitespace=true,
  commentstyle=\color{porange},
  keywordstyle=\color{pblue},
  stringstyle=\color{pgreen},
  basicstyle=\ttfamily,
  moredelim=[il][\textcolor{pgrey}]{$ $},
  moredelim=[is][\textcolor{pgrey}]{\%\%}{\%\%}
}
\newcommand{\ptitle}[1]{\underline{\textbf{#1}}}

\newenvironment{codebox} {\small \ttfamily \obeylines \begingroup \setstretch{-2.4}} {\endgroup}

% COmpletar titulo
\title{Auxiliar 9 - Hashing}
\course[CC3001]{Algoritmos y Estructuras de Datos}
\professor{Nelson Baloian}
\professor{Patricio Poblete}
\assistant{Gabriel Azócar, Manuel Cáceres}
\assistant{Michel Llorens, Sergio Peñafiel}


\begin{document}
\maketitle

\vspace{-1ex}


\begin{problems}


\problem \ptitle{Cuckoo Hashing}\\
Los Cuckoo o Cucúlidos son una especie muy particular de aves parasitarias, las cuales irrumpen en nidos ajenos para desechar los huevos de dicho nido y reemplazarlos con los propios. El 2001 Pagh \& Rodler usaron este mismo principio para formular el \textbf{Cuckoo Hashing}, el cual utiliza internamente dos tablas y dos funciones de Hashing (una para cada una de ellas) en un procedimiento donde al insertar un elemento en un casillero, si existe algo previamente allí, este elemento es ``pateado'' a la siguiente tabla y así sucesivamente. 

Con esta información:
\begin{enumerate}[a)]
    \item Implemente el método \texttt{public boolean find(int value)}.
    \item Implemente el método \texttt{public void delete(int value)}.
    \item Comente qué problemas conlleva inserción y qué se puede hacer al respecto.
    \item Implemente el método \texttt{public void insert(int value)}.
\end{enumerate}


\problem \ptitle{Hashing con prioridad}\\
%% Micho: Pronto subo la solución :D. Dudas me hacen ping!
Dependiendo de la implementación de las funciones de hashing, estas ocasionalmente llevan a tener más de un elemento en una única casilla, resultando en que buscar un elemento pase de $O(1)$ a $O(n)$, donde el peor caso es que todos los elementos caigan en la misma casilla y se deba buscar entre ellos de forma lineal. Ahora mejoraremos el costo promedio del peor caso.
\begin{enumerate}
    \item Implemente el TDA \textbf{Priority Queue} con los métodos \texttt{public void insert(Object element, String key)} y \texttt{public Object find(String key)}, donde este último aumente la prioridad del elemento buscado.
    \item Utilizando Priority Queue, implemente Hashing con encadenamiento a fin de reducir el coste promedio del peor caso.
\end{enumerate}

\end{problems}
\end{document}
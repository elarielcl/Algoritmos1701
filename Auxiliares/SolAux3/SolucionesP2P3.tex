\documentclass[dcc,uchile,sol]{fcfmcourse}
\usepackage{teoria}
\usepackage[utf8x]{inputenc}
\usepackage{amsmath}
\usepackage{amsfonts,setspace}
\usepackage{listings}
\usepackage{hyperref}
\usepackage{color}
\usepackage{tikz}





\definecolor{pblue}{rgb}{0.13,0.13,1}
\definecolor{pgreen}{rgb}{0,0.5,0}
\definecolor{porange}{rgb}{0.9,0.5,0}
\definecolor{pgrey}{rgb}{0.46,0.45,0.48}

\lstset{language=Java,
  showspaces=false,
  showtabs=false,
  breaklines=true,
  showstringspaces=false,
  breakatwhitespace=true,
  commentstyle=\color{porange},
  keywordstyle=\color{pblue},
  stringstyle=\color{pgreen},
  basicstyle=\ttfamily,
  moredelim=[il][\textcolor{pgrey}]{$ $},
  moredelim=[is][\textcolor{pgrey}]{\%\%}{\%\%}
}

\newenvironment{codebox} {\small \ttfamily \obeylines \begingroup \setstretch{-2.4}} {\endgroup}

\newcommand{\ptitle}[1]{\underline{\textbf{#1}}}

\title{Auxiliar 3 - Pauta}
\course[CC3001]{Algoritmos y Estructuras de Datos}
\professor{Nelson Baloian}
\professor{Patricio Poblete}
\assistant{Gabriel Azócar, Manuel Cáceres}
\assistant{Michel Llorens, Sergio Peñafiel}


\begin{document}
\maketitle

\vspace{-1ex}


\begin{problems}
\problem \ptitle{.}

\problem \ptitle{Ecuaciones de Recurrencia}

\begin{enumerate}
    \item $a_n = 3a_{n-1} + 4a_{n-2}$ con $a_0 = 0$ y $a_1 = 5$ 
    
    Ecuación de recurrencia lineal homogenea, luego se propone una solución del tipo:  $a_n = \lambda^n$ entonces:
    \begin{align*}
    \lambda^n = 3\lambda^{n-1} + 4\lambda^{n-2} \\
    \lambda^2 = 3\lambda + 4 \\
    (\lambda - 4)(\lambda + 1) = 0
    \end{align*}
    
    Luego $\lambda = 4$ o $\lambda = -1$, entonces $a_n = \alpha 4^n + \beta (-1)^n$, al reemplazar los valores iniciales:
    \begin{align*}
    \alpha + \beta = 0 \\
    4\alpha - \beta = 5 \\
    \Rightarrow \alpha = 1,\; \beta = -1
    \end{align*}
    
    Entonces $a_n = 4^n - (-1)^n$\\
    
    
    \item $a_n = a_{n-1} + n$ con $a_0 = 1$
    
    Al desenrollar la ecuación se obtiene:
    \begin{align*}
    a_n = a_{n-1} + n \\
    a_n = a_{n-2} + (n-1) + n \\
    a_n = a_{n-3} + (n-2) + (n-1) + n \\
    \dots
    \end{align*}
    
    El caso general es:
    
    \begin{align*}
    a_n = a_{n-k} + \sum_{i = n-k+1}^{n} i 
    \end{align*}
    
    Tomando $k = n$, y reemplazando el caso base se tiene
    \begin{align*}
    a_n = a_0 + \sum_{i = 1}^{n} i \\
    a_n = 1 + \frac{n(n+1)}{2} \\
    a_n = \frac{n^2}{2} + \frac{n}{2} + 1
    \end{align*}
    
\end{enumerate}


\problem \ptitle{Teorema Maestro y Recurrencia}\\
Obtenga las cotas superiores para los siguientes problemas utilizando el teorema maestro.
\begin{enumerate}
    \item $T(n)=2T(\frac{n}{2})+n^3$
    \begin{align*}
        T(n)=2T(\frac { n }{ 2 } )+{ n }^{ 3 }\\
        a=2\quad b=2\quad c=3 \\
        \log _{ b }{ a } =\log _{ 2 }{ 2 } =1 \\
        \log _{ b }{ a }=1<c=3\Leftrightarrow Caso\quad3 \\ \\
        Si\quad af(\frac{n}{b})\le kf(n), k<1.\\
        Entonces\quad T(n)=\Theta(f(n)) \\
        af(\frac{n}{b})=2\frac{n^3}{8} \le kn^3, \forall k\ge \frac{1}{4} \\
        \Rightarrow T(n)=\Theta(f(n))=\Theta(n^3)
    \end{align*}
    \item $T(n)=16T(\frac{n}{4})+n^2$
    \begin{align*}
        T(n)=16T(\frac { n }{ 4 } )+n^{ 2 }\\ a=16\quad b=4\quad c=\quad 2\\ \log _{ b }{ a } =\log _{ 4 }{ 16 } =2\\ \log _{ b }{ a } =2=c=2\Leftrightarrow Caso\quad 2,\quad k=0\\ \\ Si\quad T(n)=\Theta ({ n }^{ c }{ \log { n }  }^{ k+1 })\\ \Rightarrow T(n)=\Theta ({ n }^{ 2 }\log { n } )
    \end{align*}
    \item $T(n)=7T(\frac{n}{3})+n^2$
    \begin{align*}
        T(n)=7T(\frac { n }{ 3 } )+n^{ 2 }\\ a=7\quad b=3\quad c=\quad 2\\ \log _{ b }{ a } =\log _{ 3 }{ 7 } \cong 1.8\\ \log _{ b }{ a } \cong 1.8<c=2\Leftrightarrow Caso\quad 3\\ \\ af(\frac { n }{ b } )=7\frac { n^{ 2 } }{ 9 } \le kn^{ 2 },\forall k\ge \frac { 7 }{ 9 } \\ \Rightarrow T(n)=\Theta (f(n))=\Theta (n^{ 2 })
    \end{align*}
    \item $T(n)=7T(\frac{n}{2})+n^2$
    \begin{align*}
        T(n)=7T(\frac { n }{ 2 } )+n^{ 2 }\\ a=7\quad b=2\quad c=\quad 2\\ \log _{ b }{ a } =\log _{ 2 }{ 7 } \cong 2.8\\ \log _{ b }{ a } \cong 2.8>c=2\Leftrightarrow Caso\quad 1\\ \\ \Rightarrow T(n)=\Theta ({ n }^{ \log _{ b }{ a }  })\\ \Rightarrow T(n)\cong \Theta ({ n }^{ 2.8 })
    \end{align*}
    \item $T(n)=2T(\frac{n}{4})+\sqrt{n}$
    \begin{align*}
        T(n)=2T(\frac { n }{ 4 } )+\sqrt { n } \\ a=2\quad b=4\quad c=\frac { 1 }{ 2 } \\ \log _{ b }{ a } =\log _{ 4 }{ 2 } =\frac { 1 }{ 2 } \\ \log _{ b }{ a } =\frac { 1 }{ 2 } =c=\frac { 1 }{ 2 } \Leftrightarrow Caso\quad 2,\quad k=0\\ \Rightarrow T(n)=\Theta (\sqrt { n } \log { n } )
    \end{align*}
    \item $T(n)=3T(\frac{n}{2})+nlgn$
    \begin{align*}
        T(n)=3T(\frac { n }{ 2 } )+nlgn\\ a=3\quad b=2\quad c=1\\ \log _{ b }{ a } =\log _{ 2 }{ 3 } \cong 1.6\\ \log _{ b }{ a } \cong 1.6>c=1\Leftrightarrow Caso\quad 1\\ \\\Rightarrow T(n)=\Theta ({ n }^{ \log _{ b }{ a }  })\\ \Rightarrow T(n)\cong \Theta ({ n }^{ 1.6 })
    \end{align*}
    \item $T(n)=4T(\frac{n}{2})+n^2\sqrt{n}$
    \begin{align*}
        T(n)=4T(\frac { n }{ 2 } )+n^{ 2 }\sqrt { n } \\ T(n)=4T(\frac { n }{ 2 } )+n^{ \frac { 5 }{ 2 }  }\\ a=4\quad b=2\quad c=\frac { 5 }{ 2 } \\ \log _{ b }{ a } =\log _{ 2 }{ 4 } =2\\ \log _{ b }{ a } =2<c=2.5\Leftrightarrow Caso\quad 3\\ \\ af(\frac { n }{ b } )=\frac { { n }^{ 2.5 } }{ \sqrt { 2 }  } \le kn^{ 2.5 },\forall k\ge \frac { 1 }{ \sqrt { 2 }  } \\ \Rightarrow T(n)=\Theta (f(n))=\Theta (n^{ 2.5 })
    \end{align*}
    \item $T(n)=T(9\frac{n}{10})+n$
    \begin{align*}
        T(n)=T(9\frac { n }{ 10 } )+n\\ a=1\quad b=\frac { 10 }{ 9 } \quad c=1\\ \log _{ b }{ a } =\log _{ \frac { 10 }{ 9 }  }{ 1 } =0\\ \log _{ b }{ a } =0<c=1\Leftrightarrow Caso\quad 3\\ \\ af(\frac { n }{ b } )=\frac { 9n }{ 10 }\le kn, \forall k\ge \frac{9}{10}\\
        \Rightarrow T(n)=\Theta(f(n))=\Theta(n)
    \end{align*}
    \item $T(n)=2T(\frac{n}{2})+nlgn$
    \begin{align*}
        T(n)=2T(\frac { n }{ 2 } )+nlgn\\ a=2\quad b=2\quad c=1\\ \log _{ b }{ a } =\log _{ 2 }{ 2 } =1\\ \log _{ b }{ a } =1=c=1\Leftrightarrow Caso\quad 2,\quad k=1\\ \Rightarrow T(n)=\Theta (n{ \log { n }  }^{ 2 })
    \end{align*}
\end{enumerate}


\end{problems}
\end{document}

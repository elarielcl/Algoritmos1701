\documentclass[dcc,uchile]{fcfmcourse}
\usepackage{teoria}
\usepackage[utf8x]{inputenc}
\usepackage{amsmath}
\usepackage{amsfonts,setspace}
\usepackage{listings}
\usepackage{hyperref}
\usepackage{color}





\definecolor{pblue}{rgb}{0.13,0.13,1}
\definecolor{pgreen}{rgb}{0,0.5,0}
\definecolor{porange}{rgb}{0.9,0.5,0}
\definecolor{pgrey}{rgb}{0.46,0.45,0.48}

\lstset{language=Java,
  showspaces=false,
  showtabs=false,
  breaklines=true,
  showstringspaces=false,
  breakatwhitespace=true,
  commentstyle=\color{porange},
  keywordstyle=\color{pblue},
  stringstyle=\color{pgreen},
  basicstyle=\ttfamily,
  moredelim=[il][\textcolor{pgrey}]{$ $},
  moredelim=[is][\textcolor{pgrey}]{\%\%}{\%\%}
}

\newenvironment{codebox} {\small \ttfamily \obeylines \begingroup \setstretch{-2.4}} {\endgroup}

% COmpletar titulo
\title{Auxiliar 1 - ``Write Once, Run Anywhere"}
\course[CC3001]{Algoritmos y Estructuras de Datos}
\professor{Nelson Baloian}
\professor{Patricio Poblete}
\assistant{Gabriel Azócar, Manuel Cáceres}
\assistant{Michel Llorens, Sergio Peñafiel}


\begin{document}
\maketitle

\vspace{-1ex}


\begin{problems}
%Bienvenida
%Instalación java y descripción general (orientación del lenguaje)
%¿Donde aprender java?

\problem \underline{\textbf{Ordenar 3}}\\
Escriba un fragmento de código en Java que tome los valores de 3 variables enteras $a, b, c$ y deje el menor valor en $a$, el mayor valor en $c$ y el otro en $b$.
%Acá se pueden explicar ifs, else, print, que el llamado de las funciones es por valor, e "intercambiar" que será un procedimiento que utilizaremos bastante

\problem \underline{\textbf{Rotación Circular}}\\
Un string $r$ es una ``\textit{rotación circular}" de otro string $w$ si al hacer un ``shift circular" a $r$ resulta $w$.

Por ejemplo, \textit{kerstic} es una rotación cirular de \textit{sticker} y viceversa. 
Escriba la función en Java \texttt{static boolean esRotacionCircular(String r, String w)} que recibe dos strings y retorna \texttt{true} si uno es rotación circular del otro (\texttt{false} en caso contrario).
%Se puede usar el problema como escusa para mostrar la documentación de java de String y de pasada mostrar que existe la documentación), para mostrar métodos de strings, mostrar inmutabilidad de algunos objetos (String no pueden mutar)


%Problema de las fracciones/racionales
%\problem \underline{\textbf{Fracción}}\\
%Java es un lenguaje de programación orientado a %objetos, por tanto, a diferencia de lenguajes como Python el scripting no es directo y uno suele encapsular todo en clases, las cuales suelen funcionar como pequeñas máquinas a las cuales se les pide realizar acciones.

%Dicho lo anterior, se pide escribir la clase \texttt{Fraccion}, la cual para crearse necesitará tanto el numerador como el denominador y debe permitir que se simplifique o se sume a otra fracción, es decir se pueda ejecutar lo siguiente:

%\begin{lstlisting}
%> Fraccion frac = new Fraccion(10, 5);
%> Fraccion frac2 = new Fraccion(5, 5);
%> frac.suma(frac2);
%>> "3/1"
%\end{lstlisting}

%¿Cómo implementaría usted que la clase \texttt{Fraccion} pueda ser inicializada con un string? Por ejemplo \texttt{new Fraccion("10/2")}.
\problem \underline{\textbf{Fracciones}}\\
Cree la clase \texttt{Fraccion} con la siguiente definición:

\begin{enumerate}
\item Cree un constructor \texttt{public Fraccion(int num, int den)}, que inicializa una fracción usando num y den como numerador y denominador.
\item Cree otro constructor\footnote{Las clases en Java pueden tener más de un constructor siempre que reciban parámetros distintos} \texttt{public Fraccion(String frac)} donde frac es un string que representa una fracción por ejemplo ''3/5'', debe descomponer este string y guardar los numeros como enteros. (Puede suponer que el string siempre será válido).
\item Cree la función \texttt{public static int mcd(int a, int b)} que calcula el mcd entre 2 números, para esto use el algoritmo de Euclides este dice que el $mcd(a,b)=mcd(b,a\%b)$ además como caso base se tiene que $mcd(a,0)=a$.
\item Cree el método \texttt{public void simplificar()} que simplifica la fracción, para esto use \texttt{mcd}
\item Cree el método \texttt{public Fraccion suma(Fraccion other)} que devuelve la fracción simplificada que representa la suma de la fracción actual con la fracción other.
\item Cree el método \texttt{public String toString()} que devuelve un string que representa la fracción por ejemplo ''4/7''.
\end{enumerate}
\newpage
Utilice la clase anterior en un programa interactivo que sume n fracciones. Primero deberá pedir al usuario un numero n, y luego deberá pedir que ingrese n fracciones, al final deberá mostrar la suma de las n Fracciones.

\begin{codebox}
n?
3
Fraccion 1?
1/4
Fraccion 2?
1/3
Fraccion 3?
-1/12
La suma total es: 1/2
\end{codebox}


%Problema de archivos
\problem \underline{\textbf{Fibonacci}}\\
Generalmente Fibonacci se suele representar en función de sus elementos anteriores de la forma $f(n) = f(n-1) + f(n-2)$, el problema es que a medida que se requiere un número más grande, más pasos previos se requieren. Por tanto se propone ir guardando los números ya calculados en un archivo.

Teniendo ya un archivo con los primeros $n \ge 2$ números de Fibonacci, uno en cada línea, escriba la función \texttt{static void fileFibonacci()} que despliegue en pantalla los números ya calculados y además agregue al final de este el siguiente. ¿Cómo extendería esto para calcular los siguientes $k$ números en vez de sólo uno?
%La idea es explicar tanto la lectura como la escritura de archivos utilizando Buffers, así como los cast por medio de funciones y no usando el autobox (ej: String.valueOf() o Integer.parseInt() ). El problema se puede extender a por qué los números más grandes deben guardarse en forma de texto y no en variables numéricas que conllevan overflow (ver problema año 2038)
\end{problems}



\end{document}



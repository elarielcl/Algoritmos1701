\documentclass[dcc,uchile,sol]{fcfmcourse}
\usepackage{teoria}
\usepackage[utf8x]{inputenc}
\usepackage{amsmath}
\usepackage{amsfonts,setspace}
\usepackage{listings}
\usepackage{hyperref}
\usepackage{color}
\usepackage{tikz}



\definecolor{pblue}{rgb}{0.13,0.13,1}
\definecolor{pgreen}{rgb}{0,0.5,0}
\definecolor{porange}{rgb}{0.9,0.5,0}
\definecolor{pgrey}{rgb}{0.46,0.45,0.48}

\lstset{language=Java,
  showspaces=false,
  showtabs=false,
  breaklines=true,
  showstringspaces=false,
  breakatwhitespace=true,
  commentstyle=\color{porange},
  keywordstyle=\color{pblue},
  stringstyle=\color{pgreen},
  basicstyle=\ttfamily,
  moredelim=[il][\textcolor{pgrey}]{$ $},
  moredelim=[is][\textcolor{pgrey}]{\%\%}{\%\%}
}

\newenvironment{codebox} {\small \ttfamily \obeylines \begingroup \setstretch{-2.4}} {\endgroup}

\newcommand{\ptitle}[1]{\underline{\textbf{#1}}}
%%% PROPUESTAS DE EJERCICIOS
%%%% - Implementar MTF
%%%% - Un problema que se resuelva con colas de prioridad
%%%% - Implementar busqueda doblada
%%%% - Quack
%%%% - Implementar las colas de prioridad con heaps como TDA
%%%% - Problema de lista circular y eliminar kesimo
%%%% - Problema de saber si x es mayor que un k-esimo en O(k)
%%%% Funcion isMinHeap que retorna true si el heap es minHeap

\title{Auxiliar 6 - TDAs}
\course[CC3001]{Algoritmos y Estructuras de Datos}
\professor{Nelson Baloian}
\professor{Patricio Poblete}
\assistant{Gabriel Azócar, Manuel Cáceres}
\assistant{Michel Llorens, Sergio Peñafiel}


\begin{document}
\maketitle

\vspace{-1ex}


\begin{problems}
\problem \ptitle{Quack}\\
El TDA \texttt{Quack} implementa las operaciones vistas tanto para una Cola (Queue) como para una Pila (Stack), las que incluyen \texttt{push}, \texttt{pop} y \texttt{encolar}. Implemente el TDA \texttt{Quack} usando una lista enlazada, agregue además los métodos \texttt{estaVacia} y \texttt{tamano}.

\problem \ptitle{Implementación de Heaps}\\
En cátedras se vio que un \textit{maxHeap} corresponde a un árbol binario en el que el valor de sus nodos es mayor a cualquiera de sus descendientes. Además se mostró como hacer las operaciones de inserción y extracción del \textit{maxHeap}. Implemente ahora un \textit{minHeap} junto a sus operaciones, utilice un arreglo de tamaño $n$ (por lo tanto el \textit{minHeap} puede aceptar a lo más $n-1$ elementos).

\problem \ptitle{k-Mayor}\\
Escriba la función \texttt{static public boolean mayor(int[] minHeap, int k, int x)}, que retorne \texttt{true} si x es mayor que los k elementos m\'as chicos en el heap. Su función debe hacer $\mathcal{O}(k)$ comparaciones.\\
\textbf{Hint:} Simplemente recorra el heap.

\problem \ptitle{Niveles de un árbol}\\
Escriba la función \texttt{static public void imprimirNiveles(Nodo raiz)} tal que sin utilizar recursión se pueda ir imprimiendo cada nivel de un árbol.\\
\textbf{Hint:} Utilice un TDA Cola.

\end{problems}
\end{document}


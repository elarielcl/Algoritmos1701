\documentclass[dcc,uchile]{fcfmcourse}
\usepackage{teoria}
\usepackage[utf8x]{inputenc}
\usepackage{amsmath}
\usepackage{amsfonts,setspace}
\usepackage{listings}
\usepackage{hyperref}
\usepackage{color}
\usepackage[lined,boxed,commentsnumbered]{algorithm2e}
\SetKwInput{KwIn}{Input}
\renewcommand{\algorithmcfname}{Algoritmo}





\definecolor{pblue}{rgb}{0.13,0.13,1}
\definecolor{pgreen}{rgb}{0,0.5,0}
\definecolor{porange}{rgb}{0.9,0.5,0}
\definecolor{pgrey}{rgb}{0.46,0.45,0.48}

\lstset{language=Java,
  showspaces=false,
  showtabs=false,
  breaklines=true,
  showstringspaces=false,
  breakatwhitespace=true,
  commentstyle=\color{porange},
  keywordstyle=\color{pblue},
  stringstyle=\color{pgreen},
  basicstyle=\ttfamily,
  moredelim=[il][\textcolor{pgrey}]{$ $},
  moredelim=[is][\textcolor{pgrey}]{\%\%}{\%\%}
}
\newcommand{\ptitle}[1]{\underline{\textbf{#1}}}

\newenvironment{codebox} {\small \ttfamily \obeylines \begingroup \setstretch{-2.4}} {\endgroup}

% COmpletar titulo
\title{Auxiliar 13 - Búsqueda y Compresión de Texto}
\course[CC3001]{Algoritmos y Estructuras de Datos}
\professor{Nelson Baloian}
\professor{Patricio Poblete}
\assistant{Gabriel Azócar, Manuel Cáceres}
\assistant{Michel Llorens, Sergio Peñafiel}


\begin{document}
\maketitle

\vspace{-1ex}


\begin{problems}


\problem \ptitle{Búsqueda en texto (KMP)}

Escriba la función sufijoMasLargo que toma un patrón de largo m y un texto de largo n y retorna el largo máximo de un sufijo del patrón contenido en el texto.
La función debe modificar el algoritmo de Knuth − Morris − Pratt, por lo que se acepta una complejidad de O(n + m)

\problem \ptitle{Otro Algoritmo de Compresión}\\
Considere el siguiente algoritmo de compresión de un texto $T$ de largo $n$.\\
\begin{algorithm}[H]
\SetAlgoLined
\KwIn{$T[1:n]$}
 $i\gets 1$\;
 \While{$i \le n$}{
  Encontrar $T[i:j]$ más largo que aparezca como substring de $T[1:j-1]$. Digamos que este substring es $T[x:y] = T[i:j]$\;
  \texttt{print} $(x,|T[i:j]|, T[j+1])$\;
  $i\gets j+2$
 }
\end{algorithm}
\begin{enumerate}[a)]
    \item ¿Cuál es el resultado de ejecutar el algoritmo sobre $abaababbaab$?
    \item ¿Cuál es el resultado de ejecutar el algoritmo sobre $aaaaaaaaaaa$?
    \item ¿Por qué es importante $T[j+1]$ en el algoritmo? ¿Cómo soluciona LZW este fenómeno?
    \item ¿Qué problema tiene el algoritmo cuando $n$ es muy grande? ¿Cómo lo mejoraría? ¿Cuál es el costo en este caso?
\end{enumerate}
\end{problems}
\end{document}
\documentclass[dcc,uchile,sol]{fcfmcourse}
\usepackage{teoria}
\usepackage[utf8x]{inputenc}
\usepackage{amsmath}
\usepackage{amsfonts,setspace}
\usepackage{listings}
\usepackage{hyperref}
\usepackage{color}
\usepackage{tikz}





\definecolor{pblue}{rgb}{0.13,0.13,1}
\definecolor{pgreen}{rgb}{0,0.5,0}
\definecolor{porange}{rgb}{0.9,0.5,0}
\definecolor{pgrey}{rgb}{0.46,0.45,0.48}

\lstset{language=Java,
  showspaces=false,
  showtabs=false,
  breaklines=true,
  showstringspaces=false,
  breakatwhitespace=true,
  commentstyle=\color{porange},
  keywordstyle=\color{pblue},
  stringstyle=\color{pgreen},
  basicstyle=\ttfamily,
  moredelim=[il][\textcolor{pgrey}]{$ $},
  moredelim=[is][\textcolor{pgrey}]{\%\%}{\%\%}
}

\newenvironment{codebox} {\small \ttfamily \obeylines \begingroup \setstretch{-2.4}} {\endgroup}

\newcommand{\ptitle}[1]{\underline{\textbf{#1}}}

\title{Auxiliar 3 - Recursión y Recurrencias}
\course[CC3001]{Algoritmos y Estructuras de Datos}
\professor{Nelson Baloian}
\professor{Patricio Poblete}
\assistant{Gabriel Azócar, Manuel Cáceres}
\assistant{Michel Llorens, Sergio Peñafiel}


\begin{document}
\maketitle

\vspace{-1ex}


\begin{problems}
\problem \ptitle{Invertir}\\
Para lo siguiente utilice recursión:
\begin{enumerate}[a)]
\item Escriba la función \texttt{static int pow(int a, int N)} que retorna $a^N$
\item Escriba la función \texttt{static int log(int a, int N)} que retorna $\lfloor \log_{a}N \rfloor$
\item Escriba la función \texttt{static int invertir(int N)} que retorna N con sus dígitos invertidos
\end{enumerate}


\problem \ptitle{Ecuaciones de recurrencia}\\
Resuelva la siguientes ecuaciones de recurrencia:

\begin{enumerate}[a)]
    \item $a_n = 3a_{n-1} + 4a_{n-2}$ con $a_0 = 0$ y $a_1 = 5$ 
    %Solución: a_n = 4^n - (-1)^n
    \item $a_n = a_{n-1} + n$ con $a_0 = 1$ \\
    \textbf{Propuesto:} considere ahora $a_n = ka_{n-1} + n, \; a_0 = 1$. ¿Cambia el orden de la solución obtenida si $k>1$?
\end{enumerate}

\problem \ptitle{Teorema Maestro y Recurrencia}\\
Obtenga las cotas superiores para los siguientes problemas utilizando el teorema maestro.
\begin{enumerate}[a)]
    \item $T(n)=2T(\frac{n}{2})+n^3$
    \item $T(n)=16T(\frac{n}{4})+n^2$
    \item $T(n)=7T(\frac{n}{3})+n^2$
    \item $T(n)=7T(\frac{n}{2})+n^2$
    \item $T(n)=2T(\frac{n}{4})+\sqrt{n}$
    \item $T(n)=3T(\frac{n}{2})+nlgn$
    \item $T(n)=4T(\frac{n}{2})+n^2\sqrt{n}$
    \item $T(n)=T(9\frac{n}{10})+n$
    \item $T(n)=2T(\frac{n}{2})+nlgn$
\end{enumerate}


\end{problems}
\end{document}

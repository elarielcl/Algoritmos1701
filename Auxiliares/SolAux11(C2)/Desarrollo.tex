
 \documentclass[dcc,sol]{fcfmcourse}
\usepackage{teoria}
\usepackage[utf8x]{inputenc}
\usepackage{amsmath}
\usepackage{amsfonts,setspace}
\usepackage{listings}
\usepackage{color}

\usepackage{tikz}

\usepackage{verbatim}
\usetikzlibrary{arrows,shapes}

\definecolor{pblue}{rgb}{0.13,0.13,1}
\definecolor{pgreen}{rgb}{0,0.5,0}
\definecolor{porange}{rgb}{0.9,0.5,0}
\definecolor{pgrey}{rgb}{0.46,0.45,0.48}

\lstset{language=Java,
  showspaces=false,
  showtabs=false,
  breaklines=true,
  showstringspaces=false,
  breakatwhitespace=true,
  commentstyle=\color{porange},
  keywordstyle=\color{pblue},
  stringstyle=\color{pgreen},
  basicstyle=\ttfamily,
  moredelim=[il][\textcolor{pgrey}]{$ $},
  moredelim=[is][\textcolor{pgrey}]{\%\%}{\%\%}
}

\newenvironment{codebox} {\small \ttfamily \obeylines \begingroup \setstretch{-2.4}} {\endgroup}

\title{Auxiliar 11 - Preparación Control 2}
\course[CC3001]{Algoritmos y Estructuras de Datos}
\professor{Nelson Baloian}
\professor{Patricio Poblete}
\assistant{Gabriel Azócar, Manuel Cáceres}
\assistant{Michel Llorens, Sergio Peñafiel}


\begin{document}
\maketitle

\vspace{-1ex}


\begin{problems}

\problem
\problem El costo de utilizar HeapSort en $n$ elementos es el de construcción (lineal) y de extracción de elementos (linearítmico)
\begin{align*}
    \mathcal{O}(n + n\log n) &= \mathcal{O}(n\log n)
\end{align*}
\section*{Idea}
\begin{itemize}
    \item  Dividir el arreglo en $\approx \sqrt{n}$ de tamaño $\approx \sqrt{n}$
    \item Ordenar con HeapSort cada una de esas partes a costo
    \begin{align*}
        \sqrt{n}\cdot \mathcal{O}(\sqrt{n} \log{\sqrt{n}}) = \mathcal{O}(n\log{\sqrt{n}}) = \mathcal{O}(n\log n)
    \end{align*}
\end{itemize}
\begin{enumerate}[a)]
    \item Hacer merge de las $\sqrt{n}$ partes buscando el mínimo secuencialmente a costo
    \begin{align*}
        n \cdot \sqrt{n} = n^{3/2}
    \end{align*}
    , y por lo tanto, el costo total es $\mathcal{O}(n\log n + n ^{3/2}) = \mathcal{O}(n^{3/2})$
    \item Hacer merge de las $\sqrt{n}$ partes manteniendo un heap que se forma de los mínimos de cada una de las partes.\\
    
    Los $n$ elementos deben ser insertados y eliminados de un heap de tamaño $\sqrt{n}$ a lo más una vez, por lo que su costo es $2n\log{\sqrt{n}} = n\log n$ y el costo total es $\mathcal{O}(n\log n + n\log n) = \mathcal{O}(n\log n)$
\end{enumerate}
\problem
\end{problems}
\end{document}

